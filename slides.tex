\documentclass[mathserif]{beamer}%
%
%
\RequirePackage{./styles/slides}%
%
\title{Example for Slides}%
%
\blAuthor{Thomas Weise}{汤卫思}{tweise@hfuu.edu.cn}{http://iao.hfuu.edu.cn}%
%
\blInstitute{%
\blUni{Hefei University, South Campus}{合肥学院 南艳湖校区}\\%
\blFaculty{Faculty of Computer Science and Technology}{计算机科学与技术系}\\%
\blGroup{Institute of Applied Optimization}{应用优化研究所}\\%
\blAddressA{230601}{Hefei, Anhui, China}{中国 安徽省 合肥市}\\%
\blAddressB{Econ.\ \& Tech.\ Devel.\ Zone, Jinxiu Dadao 99}{经济技术开发区 锦绣大道99号}%
}%
%
\date{\today}%
%
%
\begin{document}%
%
\startPresentation{}%
%
\AtBeginSection[]{%
\begin{frame}%
\frametitle{Section Outline}%
\tableofcontents[currentsection]%
\end{frame}%
}%
%
%
\section{Introduction}%
%
\begin{frame}%
\frametitle{\LaTeX}%
\begin{itemize}%
\item This is a theme for creating nice slides with \LaTeX\scitep{GMS1994TLC,L1994LADPSUGARM,MGBCR2004TLC,OPHS2011TNSSITLOLI1M} and the beamer document class%
\end{itemize}%
\end{frame}%
%
\begin{frame}%
\frametitle{Structure}%
\begin{itemize}%
\item A presentation is structured into%
\begin{enumerate}%
\item a head, where you write information such as the title and your affiliation,%
\item a body, with the actual contents, and%
\item a foot, with the references and the good-bye slide%
\end{enumerate}%
\item The body can and should be divided into sections with the \texttt{{\textbackslash}section{\dots}} command%
\item Each section can contain multiple \texttt{frame}s, where each frame is one slide (which may be composed of several steps, see later)%
\end{itemize}%
\end{frame}%
%
\section{\LaTeX\ Commands}%
%
\begin{frame}%
\frametitle{Itemizations}%
This is an introduction to itemizations.%
\begin{itemize}%
\item We can have item lists with the \texttt{itemize} environment.%
\item Each item will then begin with an \texttt{{\textbackslash}item} command.%
\item Itemizations can also be nested:%
\begin{itemize}%
\item Like this.%
\item Which is nice too.
\end{itemize}%
\end{itemize}%
\end{frame}%
%
\begin{frame}%
\frametitle{Enumerations}%
This is an introduction to enumerations.%
\begin{enumerate}%
\item We can have numbered lists with the \texttt{enumerate} environment.%
\item Each item will then begin with an \texttt{{\textbackslash}item} command.%
\item Enumerations can also be nested:%
\begin{enumerate}%
\item Like this.%
\item Which is nice too.
\end{enumerate}%
\item We can also nest itemizations in them, and vice versa:%
\begin{itemize}%
\item See?%
\item And now another enumeration:%
\begin{enumerate}%
\item See?
\end{enumerate}%
\end{itemize}%
\end{enumerate}%
\end{frame}%
%
\begin{frame}[t]%
\frametitle{Showing Stuff Step-Wise}%
Things can be shown step-wise\uncover<2->{:%
\begin{itemize}%
\item If you want to show a thing on the $n$\textsuperscript{th} \inQuotes{step} of a slide, you can wrap it into a command of the form \texttt{{\textbackslash}uncover<$n$->\{\dots\}}%
\item<3-> It then stays invisible until the $n$\textsuperscript{th} step is reached, but occupies space.%
\item<4-> If you want something to remain invisible and not occupy space, you can use \texttt{{\textbackslash}only<$n$->\{\dots\}}%
\item<5-> Actually, the stuff inside the \texttt{<\dots>} marks a range of steps.\only<6-9>{ Write%
\begin{itemize}%
\item \texttt{<10>} to show something \emph{only} at the 10\textsuperscript{th} step\uncover<7->{,}%
\item<7-> \texttt{<10->} to show something from at the 10\textsuperscript{th} step onwards\uncover<8->{,}%
\item<8-> \texttt{<9-11>} to show something from the 9\textsuperscript{th} to the 11\textsuperscript{th} step\uncover<9->{,}%
\item<9-> \texttt{<4,9-11,14->} to show something from at the 4\textsuperscript{th}, the 9\textsuperscript{th} to the 11\textsuperscript{th}, and from the 14\textsuperscript{th} step onwards, and so on.%
\end{itemize}}%
\only<10->{%
\item<10-> Many commands, such as \texttt{{\textbackslash}item}, can take an argument of the form \texttt{<...>}, which makes their effect similar to be wrapped into an \texttt{{\textbackslash}uncover<$n$->\{\dots\}}.%
\only<11->{%
\item<11-> If you use \texttt{{\textbackslash}only} in your slides, you better use \texttt{{\textbackslash}begin\{frame\}[t]} instead of \texttt{{\textbackslash}begin\{frame\}} to begin a frame, or stuff will jump around like on this slide here.%
}}%
%
\end{itemize}%
}%
\end{frame}%
%
\begin{frame}%
\frametitle{Colors}%
\begin{itemize}%
\item You can use colors to color something%
\item<2-> You would use the \texttt{{\textbackslash}textcolor\{color\}\{text\}} command for this and, e.g., write \textcolor{red}{\texttt{{\textbackslash}textcolor\{red\}\{hello\}}}%
\item<3-> This command also takes an \texttt{<\dots>} argument, which allows you to do \textcolor<4,6>{green}{\texttt{{\textbackslash}textcolor<4,6>\{green\}\{hello\}}}%
\item<7-> You can also use the command \texttt{{\textbackslash}alert\{\dots\}} to mark \alert{something especially important}.%
\item<8-> It, too, can take the \texttt{<\dots>} argument.%
\end{itemize}%
\end{frame}%
%
\begin{frame}%
\frametitle{References}%
Commands for references to literature (stored as Bib\TeX\ records in the file \texttt{bibliography/bibliography.bib}):%
\begin{itemize}%
\item\small numerical citations are done with \texttt{{\textbackslash}citep\{reference\}} and look like \inQuotes{blabla\citep{WGOEB}}%
\item\small numerical citations with non-breakable space in front are done with \texttt{{\textbackslash}scitep\{reference\}} and look like \inQuotes{blabla\scitep{WWCTL2016GVLSTIOPSOEAP}}, which is usually the better way%
\item\small citations with author names and numerical id are done as \texttt{{\textbackslash}citet\{reference\}} and look like \inQuotes{blabla \citet{WNSRG2008ATMFMOERANFL}}%
\item\small citations with author names and numerical id at the beginning of a sentence are done as \texttt{{\textbackslash}Citet\{reference\}} and look like \inQuotes{blabla. \Citet{WPG2009SRWVRPWEA}}%
\item\small citations with author names only are done as \texttt{{\textbackslash}citeauthor\{reference\}} and look like \inQuotes{blabla \citeauthor{WCTLTCMY2014BOAAOSFFTTSP}}%
\item\small citations with author names only at the beginning of a sentence are done as \texttt{{\textbackslash}Citeauthor\{reference\}} and look like \inQuotes{blabla. \Citeauthor{WCT2012EOPABT}}%
\end{itemize}%
\end{frame}%
%
\begin{frame}[t]%
\frametitle{Positioning of Graphics}%
\begin{itemize}%
\item graphics can be sized and positioned using commands such as %
\only<-2>{\texttt{{\textbackslash}locateGraphic\{when\}\{arg\}\{path\}\{x\}\{y\}}}%
\only<3>{\texttt{{\textbackslash}locateGraphic[taken from{\textbackslash}scitep\{WGOEB\}]\{when\}\{arg\}\{path\}\{x\}\{y\}}}%
\only<4>{\texttt{{\textbackslash}locateFramedGraphic\{when\}\{arg\}\{path\}\{x\}\{y\}}}%
\only<-1>{ where%
\begin{enumerate}%
\item[\texttt{when}] is the slide index for displaying the graphic (see \texttt{{\textbackslash}uncover} et al)%
\item[\texttt{arg}] is an argument passed to \LaTeX, e.g., \texttt{width=0.5{\textbackslash}paperwidth} indicates a graphic scaled to 50\% of the slide width%
\item[\texttt{path}] is the relative path to the graphic%
\item[\texttt{x}] is the \texttt{x}-coordinate (0 for left end, 1 for right end)%
\item[\texttt{y}] is the \texttt{y}-coordinate (0 for top end, 1 for bottom end)%
\end{enumerate}%
}%
\end{itemize}%
\locateGraphic{2}{width=0.6\paperwidth}{graphics/iao/iao_slides_bibliography/iao_slides_bibliography}{0.2}{0.4}%
\locateGraphic[taken from\scitep{WGOEB}]{3}{width=0.6\paperwidth}{graphics/iao/iao_slides_bibliography/iao_slides_bibliography}{0.2}{0.4}%
\locateFramedGraphic{4}{width=0.6\paperwidth}{graphics/iao/iao_slides_bibliography/iao_slides_bibliography}{0.2}{0.4}%
%
\end{frame}%
%
\begin{frame}%
\frametitle{Math}%
\begin{itemize}%
\item We can include mathematical content between \texttt{\$\dots\$} signs inside the text, i.e., writing \texttt{\${\textbackslash}sqrt\{5*{\textbackslash}left({\textbackslash}frac\{{\textbackslash}pi\}\{2\}-{\textbackslash}mathbb\{R\}{\textbackslash}right)\}\$} yields $\sqrt{5*\left(\frac{\pi}{2}-\mathbb{R}\right)}$
\item<2-> Bigger equations can be printed with the \texttt{equation} environment.%
\end{itemize}%
\uncover<2->{%
\begin{equation}%
\sqrt{\sum_{i=1}^n \log \left(\frac{\max\left\{i,\frac{n}{3}\right\}}{3\pi} -7e^3 \right)}%
\end{equation}}%
\end{frame}%
%
\begin{frame}%
\frametitle{Blocks}%
\begin{block}{Block Title}%
\begin{itemize}%
\item 1%
\item 2%
\end{itemize}%
\end{block}%
\end{frame}%
%
\endPresentation%
%
\end{document}%%
\endinput%
%

